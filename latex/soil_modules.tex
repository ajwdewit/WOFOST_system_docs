\chapter{Soil modules}

\section{Water balance modules}

Plant growth involves intake of atmospheric CO$_{{\rm 2}}$ through stomatal openings in the epidermis. Most of the water that plants take up from the soil is again lost to the atmosphere by transpiration through the same openings. The daily turnover can be considerable: transpiration from 0.4 cm of water from a crop surface on a clear sunny day corresponds with a water loss from the root zone of 40.000 kg ha$^{{\rm -1}}$ 
d$^{{\rm -1}}$. If soil moisture take up by the roots is not replenished, the soil will dry out to such an extent that the plants wilt and - ultimately - die.

A crop growth simulation model must therefore keep track of the soil water balance which compares for a given period of time, incoming water in the rooted soil with outgoing water and quantifies the difference between the two as a change in the amount of soil moisture stored.


\begin{figure}[p]
	%Fig. 6.1
	\centering
	\includegraphics[width=120mm]{\FigDir/WATBAL1.pdf}
	\caption{Schematic representation of the different components of a soil water balance}
	\label{fig:WatBalSchematic}
\end{figure}

The actual root zone soil moisture content can be established according to (Driessen, 1986):

\stepcounter{equation}
\begin{align}
% equations 6.1a-b
\label{eq:6.1}
theta_{t~} &= theta_{t-1~} + {\frac{IN _{up} ~+~IN _{low} ~-~T _{a}}{RD}} \Delta t \subeqn \\[1em]
IN_{up} &= P~+~I _{e} ~-~E _{s} ~+~ \delta SS_{t} \, /\Delta t-~SR  \subeqn  \\[1em]
IN_{low} &= CR~-~Perc \subeqn
\end{align}

Where:\\[5pt]
\begin{tabularx}{\textwidth}{llXr}
	$\theta$$_{{\rm t}}$ &:& Actual moisture content of the root zone at time step t
	& [cm$^{{\rm 3}}$ cm$^{{\rm -3}}$]\\
	IN$_{{\rm up}}$ &:& Rate of net influx through the upper root zone boundary
	& [cm d$^{{\rm -1}}$]\\
	IN$_{{\rm lo}}$$_{{\rm w}}$ &:& Rate of net influx through the lower root zone boundary
	& [cm d$^{{\rm -1}}$]\\
	T$_{{\rm a}}$ &:& Actual transpiration rate of crop
	& [cm d$^{{\rm -1}}$]\\
	RD &:& Actual rooting depth  & [cm]\\
	P &:& Precipitation rate  & [cm d$^{{\rm -1}}$]\\
	I$_{{\rm e}}$ &:& Effective daily irrigation  & [cm d$^{{\rm -1}}$]\\
	E$_{{\rm s}}$ &:& Soil evaporation rate   & [cm d$^{{\rm -1}}$]\\
	$\delta SS_{{\rm t}}$ &:& Surface storage  & [cm]\\
	SR &:& Rate of surface runoff  & [cm d$^{{\rm -1}}$]\\
	CR &:& Rate of capillary rise  & [cm d$^{{\rm -1}}$]\\
	Perc &:& Percolation rate  & [cm d$^{{\rm -1}}$]\\
	$\Delta$t &:& Time step  & [d]\\
%	Z$_{{\rm t}}$ &:& Depth of groundwater table (see figure \ref{fig:WatBalSchematic})  & [cm]\\
\end{tabularx}

Processes directly affecting soil moisture content of the root zone can be defined as:
\begin{itemize}
	\item infiltration is transport from the soil surface into the root zone;
	\item evaporation is the loss of soil moisture to the atmosphere through the soil surface;
	\item transpiration by plants is loss of soil moisture to the atmosphere 
	      from the entire root zone;
	\item percolation is downward transport of water from the root zone to the layer below the root zone;
	\item capillary rise is upward transport into the rooted zone.
\end{itemize}

The water balance equation has to be solved for each time interval during the crop growth cycle.

For the calculation of potential production the soil moisture content is assumed to be at field capacity and no calculations need to be performed to simulate the water balance in the soil. Only the crop water requirements are quantified as the sum of crop transpiration and evaporation from the shaded soil under the canopy. 

For the calculation of the water limited production two different soil water balances are distinguished. The first water balance (called \emph{Classic water balance}) applies to a freely draining soil, where groundwater is so deep that it cannot have influence on the soil moisture content in the rooting zone. The soil profile is divided in two compartments, the rooted zone and the lower zone between actual rooting depth and maximum rooting depth. The subsoil below the maximum rooting depth is not defined. The second zone merges gradually with the first zone as the roots grow deeper towards the maximum rooting depth. This water balance applies to (regional) applications with limited information on soil properties.

Second, the WOFOST implementation connected to the SWAP model has a detailed water balance including multiple soil layers and variable integration time steps to account for highly non-linear processes in the soil. Besides estimating water availability, SWAP also deals with soil temperature and solute transport which allows to make detailed simulations of the behaviour of water and solutes in the soil and its impact on plant growth.The SWAP water balance is not treated in this manual as SWAP has its own set of documentation \footnote{http://www.swap.alterra.nl} where the theory and usage of the model is described.

\subsection{Classic waterbalance: potential production}
\label{sec:WATPP}

The purpose of the water balance for potential production is to quantify the crop water requirements for continuous growth without drought stress. It is assumed that the soil is permanently at field capacity.

\begin{equation}
\label{eq:6.15}
\theta  _{t} ~ =~\theta  _{fc} 
\end{equation}

Where:\\[5pt]
\begin{tabularx}{\textwidth}{llXr}
	$\theta$$_{{\rm t}}$ &:& Actual soil moisture content & [cm$^{{\rm 3}}$ cm$^{{\rm -3}}$]\\
	$\theta$$_{{\rm fc}}$ &:& Soil moisture content at field capacity & [cm$^{{\rm 3}}$ cm$^{{\rm -3}}$]\\
\end{tabularx}

Rainfall, irrigation, capillary rise and drainage are not taken into account. The only two processes to consider are evaporation of the surface and transpiration of the crop. The calculation of the crop transpiration is described under section  \S \ref{sec:evapotranspiration} while evaporation is treated in the same way as in the water balance for free drainage conditions (\S \ref{sec:evaporation}).

%In the previous paragraph the
%calculation of these processes is described in detail.
%
%The evaporation rate is calculated for two situations. In the first situation
%Airducts are present in the roots. This means that the crop tolerates waterlogging. This
%is the case for example with rice. The evaporation rate is then calculated as the 
%evaporation rate from a shaded water surface:
%
%\begin{equation}
%%eq 6.16
%E _{w} ~=~ E _{w, \max } 
%\end{equation}
%
%Where:\\[5pt]
%\begin{tabularx}{\textwidth}{llXr}
%	E$_{{\rm w}}$ &:& Evaporation rate from a shaded water surface & [cm d$^{{\rm -1}}$]\\
%	E$_{{\rm w,max}}$ &:& Maximum evaporation rate from shaded 
%	water surface (see eq. \ref{eq:6.6}) & [cm d$^{{\rm -1}}$]\\
%\end{tabularx}
%
%In the second situation airducts are absent in the roots. This means that the crop does
%not tolerate water
%logging. This is the case for most crops. The evaporation rate is then calculated as the
%evaporation rate from a soil surface (van Diepen, personal communication):  
%
%\begin{equation}
%\label{eq:6.17}
%E _{s~} =~ E _{s,\max } {\frac{ \theta  _{fc} ~-~ {{\frac{\theta  _{wp} }{3}}} }{\theta  _{\max } ~-~{{\frac{\theta  _{wp} }{3}}} }}
%\end{equation}
%
%Where:\\[5pt]
%\begin{tabularx}{\textwidth}{llXr}
%	E$_{{\rm s}}$ &:& Evaporation rate from a shaded soil surface & [cm d$^{{\rm -1}}$]\\
%	E$_{{\rm s,max}}$ &:& Maximum evaporation rate from a shaded
%	soil surface (see eq. \ref{eq:6.7}) & [cm d$^{{\rm -1}}$]\\
%	$\theta$$_{{\rm fc}}$ &:& Soil moisture content at field capacity & [cm$^{{\rm 3}}$ cm$^{{\rm -3}}$]\\
%	$\theta$$_{{\rm wp}}$ &:& Soil moisture content at wilting point & [cm$^{{\rm 3}}$ cm$^{{\rm -3}}$]\\
%	$\theta$$_{{\rm max}}$ &:& Soil porosity & [cm$^{{\rm 3}}$ cm$^{{\rm -3}}$]\\
%\end{tabularx}


\subsection{Classic Water balance: water-limited production under free drainage conditions}
\label{sec:WATFD}

This soil module computes the variables of the soil water balance in the water-limited production scenario for a freely draining soil. No influence from groundwater is assumed. The purpose is to quantify the crop water use while subject tot either drought stress or water excess, and to quantify a possible reduction of the crop transpiration rate, leading to a reduced growth.

For the rooted zone the water balance equation is solved every daily time step. At the upper boundary, processes comprise the infiltration of water from precipitation or irrigation, evaporation from the soil surface and uptake of water and transpiration by the crop. If rainfall intensity exceeds the infiltration and surface storage capacity of the soil, water runs off. Water can be stored in the soil till the field capacity is reached. Additional water percolates beyond the lower boundary of the rooting zone. The flow rates are limited by the maximum percolation rate of the root zone and the maximum percolation rate of the water to the subsoil.

The textural profile of the soil is conceived homogeneous. Initially the soil profile consists of three layers (zones):
\begin{itemize}
	\item the rooted zone between soil surface and actual rooting depth
	\item the lower zone between actual rooting depth and maximum rooting depth
	\item the subsoil below maximum rooting depth
\end{itemize}

The extension of the root zone from initial rooting depth to maximum rooting depth is described in \S \ref{sec:rootgrowth}. Its effect on the soil moisture content is accounted for in this soil water balance calculation. From the moment that the maximum rooting depth is reached the soil profile is described as a two layer system (Driessen, 1986). The lower zone no longer exists.

\subsubsection{Initial soil water content and initial soil water amount}

The initial value of the actual soil moisture content in the rooted part of the soil can be
calculated as:

\begin{equation}
\label{eq:6.18}
\theta_{t} ~ =~\theta_{wp} ~+~{\frac{W_{av}}{RD}}
\end{equation}

Where:\\[5pt]
\begin{tabularx}{\textwidth}{llXr}
	$\theta_{{\rm t}}$ &:& Actual soil moisture content in rooted z
	one  & [cm$^{{\rm 3}}$ cm$^{{\rm -3}}$]\\
	$\theta_{{\rm wp}}$ &:& Soil moisture content at wilting   & [cm$^{{\rm 3}}$ cm$^{{\rm -3}}$]\\
	W$_{{\rm av}}$ &:& Initial amount of available water in the soil 
	in excess of $\theta_{{\rm wp}}$ & [cm]\\
	RD &:& Actual rooting depth (see \S 5.5) & [cm]\\
\end{tabularx}

It should be mentioned that the initial actual soil moisture content, $\theta_{{\rm t}}$ in WOFOST, cannot be lower than the soil moisture content at wilting point. In case the crop cannot develop airducts, the initial soil moisture content cannot be higher than the soil moisture content at field capacity. If the crop can develop airducts the initial soil moisture content cannot exceed the soil porosity. {\bf W$_{{\rm av}}$} (acronym: {\bf WAV}), the initial amount of available soil moisture in excess of $\theta_{{\rm wp}}$ should be provided by the user. 

Multiplying the actual soil moisture content with the rooting depth yields the initial amount of water in the rooted zone. The initial amount of soil moisture in the lower zone, the zone between the rooted zone and the maximum rooting depth, can is as:

\begin{equation}
\label{eq:6.19}
W_{lz} ~ =~ W _{av} ~+~ RD_{\max } ~\theta_{wp} ~-~RD~\theta_{t} 
\end{equation}

Where:\\[5pt]
\begin{tabularx}{\textwidth}{llXr}
	W$_{{\rm lz}}$ &:& Amount of soil moisture in the lower zone & [cm]\\
	W$_{{\rm av}}$ &:& Initial amount of available soil moisture 
	in excess of $\theta$$_{{\rm wp}}$ & [cm]\\
	RD$_{{\rm max}}$ &:& Maximum rooting depth & [cm]\\
	RD &:& Actual rooting depth & [cm]\\
	$\theta$$_{{\rm t}}$ &:& Actual soil moisture content in rooted zone  
	& [cm$^{{\rm 3}}$ cm$^{{\rm -3}}$]\\
	$\theta$$_{{\rm wp}}$ &:& Soil moisture content at wilting point  
	& [cm$^{{\rm 3}}$ cm$^{{\rm -3}}$]\\
\end{tabularx}

The soil moisture content of the lower zone is also limited by the field capacity in case the crop cannot develop airducts, else the soil moisture content is limited by the soil porosity. 

In the model, initially, the variable D$_{{\rm slr}}$, days since last rain, is set  to one. If the actual soil moisture content is halfway between the field capacity and wilting point a value of five days is assumed. 

\subsubsection{Evaporation}
\label{sec:evaporation}
The evaporation from the surface below the canopy to the atmosphere can either come from a soil surface or a water surface. It also depends on the amount of available water in the soil and the infiltration capacity of the soil. If the water layer on the surface, the so called surface storage, exceeds 1 cm, the actual evaporation rate from the soil is set to zero and the actual evaporation rate from the surface water is equal to the maximum evaporation from a shaded water surface. If the surface storage is less than 1 cm and the infiltration rate of the previous day exceeds 1 cm d$^{{\rm -1}}$, the actual evaporation rate from the surface water is set to zero and the actual evaporation rate from the soil is equal to the maximum evaporation from a shaded soil surface. 

Without water on the soil surface, the variable D$_{{\rm slr}}$, days since last rain, is use to control the decay in surface evaporation rate due to the drying out of the soil surface. The reduction of the evaporation is thought to be proportional to the square root of time (Stroosnijder, 1987, 1982). The initial value of D$_{{\rm slr}}$ depends on the moisture content in the top layer. It is set to value of five days if the soil moisture content is halfway between the field capacity and wilting point, otherwise a value of one day is used. The value of the variable days since last rain, D$_{{\rm slr}}$ increases every day thereby decreasing the actual soil evaporation rate. Its value is reset to one if the infiltration rate of the previous is larger than 1 cm d$^{{\rm -1}}$, The soil evaporation is calculated as:

\begin{equation}
\label{eq:Es}
E_{s} ~=~ E_{s,\max } ( \sqrt{D_{slr} } ~-~ \sqrt{D_{slr} \, -\, 1} )
\end{equation}

Where:\\[5pt]
\begin{tabularx}{\textwidth}{llXr}
	E$_{{\rm s}}$ &:& Evaporation rate from a shaded soil surface  & [cm d$^{{\rm -1}}$]\\
	E$_{{\rm s,max}}$ &:& Maximum evaporation rate from a shaded soil
	surface (see eq. \ref{eq:Esmax})  & [cm d$^{{\rm -1}}$]\\
	D$_{{\rm slr}}$ &:& Days since last rain  & [d]\\
\end{tabularx}

When a small amount of water has infiltrated, or rather wetted the soil surface, this
amount can be evaporated the same day, irrespective of D$_{{\rm slr}}$. Therefore, the actual
evaporation from the soil surface, as calculated according to equation \ref{eq:Es}, should be
corrected for this amount of water infiltrating the soil. This amount should be added to
the actual evaporation rate. However, it should be noted that the actual evaporation never
can exceed the maximum evaporation rate.

\subsubsection{Percolation}
If the soil moisture content of the root zone is above field capacity, water percolates to
the lower part of the potentially rootable zone and to the subsoil. In the model, a clear
distinction is made between percolation from the actual rootzone to the so-called lower
zone, and percolation from the lower zone to the subsoil. The former is called Perc and
the latter is called Loss. The percolation rate from the rooted zone is calculated as:

\begin{equation}
\label{eq:6.21}
Perc  ~=~{\frac{W _{rz} \, -\, W _{rz,\, fc} }{\Delta t}} ~-~ T _{a} ~-~ E _{s} 
\end{equation}

Where:\\[5pt]
\begin{tabularx}{\textwidth}{llXr}
	Perc &:& Percolation rate from the root zone to the lower zone  & [cm d$^{{\rm -1}}$]\\
	W$_{{\rm rz}}$ &:& Amount of soil moisture in the root zone  & [cm]\\
	W$_{{\rm rz,fc}}$ &:& Equilibrium amount of soil moisture in the root
	zone (see eq. \ref{eq:6.22})  & [cm]\\
	$\Delta$t &:& Time step  & [d]\\
	T$_{{\rm a}}$ &:& Actual transpiration rate (see eq. \ref{eq:6.14})  & [cm d$^{{\rm -1}}$]\\
	E$_{{\rm s}}$ &:& Evaporation rate from a shaded soil surface 
	(see eq. \ref{eq:Es})  & [cm d$^{{\rm -1}}$]\\
\end{tabularx}

The equilibrium amount of soil moisture in the root zone is calculated as the soil
moisture content at field capacity times the depth of the rooting zone:

\begin{equation}
\label{eq:6.22}
W _{rz,\, fc} ~=~ \theta  _{fc} \,\, RD
\end{equation}

Where:\\[5pt]
\begin{tabularx}{\textwidth}{llXr}
	W$_{{\rm rz,fc}}$ &:& Equilibrium amount of soil moisture in the root zone  & [cm]\\
	$\theta$$_{{\rm fc}}$ &:& Soil moisture content at field capacity  & 
	[cm$^{{\rm 3}}$ cm$^{{\rm -3}}$]\\
	RD &:& Actual rooting depth  & [cm]\\
\end{tabularx}

The percolation rate is limited by the conductivity of the wet soil (acronym: {\bf SOPE}) in the
same way as the infiltration is limited. The conductivity is soil specific and should be
given by the user. Note that the percolation from the root zone to the lower zone can be
limited by the uptake capacity of the lower zone. Therefore, the value calculated with 
equation \ref{eq:6.21} is preliminary. The capacity should first be checked.

The loss of water from the lower zone to the subsoil, the so-called Loss, should take the
amount of water in the lower zone into account. If the amount of water in the lower zone
is less than the equilibrium amount of soil moisture, a part of the percolating water will
be retained and the percolation rate will be reduced, thus:

\begin{equation}
\label{eq:6.23}
Loss ~=~{\frac{W _{lz} \, -\, W _{lz,\, fc} }{\Delta t}} ~+~ Perc
\end{equation}

Where:\\[5pt]
\begin{tabularx}{\textwidth}{llXr}
	Loss &:& Percolation rate from the lower zone to the subsoil   & [cm d$^{{\rm -1}}$]\\
	Perc &:& Percolation rate from root zone to lower zone (see eq. \ref{eq:6.21})  & [cm d$^{{\rm -1}}$]\\
	W$_{{\rm lz}}$ &:& Amount of soil moisture in the lower zone & [cm]\\
	W$_{{\rm lz,fc}}$ &:& Equilibrium amount of soil moisture in the
	lower zone (see eq. \ref{eq:6.24})  & [cm]\\
	$\Delta$t &:& Time step   & [d]\\
\end{tabularx}

The loss of water from the potentially rootable zone, is also limited by the maximum percolation rate of the subsoil. This maximum percolation rate (acronym: {\bf KSUB}) is soil specific and should be provided by the user. The equilibrium amount of soil moisture in the lower zone can be calculated as the soil moisture content at field capacity times the depth of the root zone:

\begin{equation}
\label{eq:6.24}
W_{lz,\, fc} = \theta_{fc} \,\, (RD_{\max} ~-~RD)
\end{equation}

Where:\\[5pt]
\begin{tabularx}{\textwidth}{llXr}
	W$_{{\rm rz,fc}}$ &:& Equilibrium amount of soil moisture in the lower zone  & [cm]\\
	$\theta$$_{{\rm fc}}$ &:& Soil moisture content at field capacity  & [cm$^{{\rm 3}}$ cm$^{{\rm -3}}$]\\
	RD$_{{\rm max}}$ &:& Maximum rooting depth  & [cm]\\
	RD &:& Actual rooting depth  & [cm]\\
\end{tabularx}

The saturated soil conductivity (acronym: {\bf K0}) is soil specific. In the model, K0 is calculated from the pF curve assuming $pF = -1.0$ (i.e. a hydraulic head of 0.1 cm) The percolation rate from the lower zone to the sub soil is not to exceed this value (van Diepen {\it et al}., 1988). 

As mentioned before, the value calculated with equation \ref{eq:6.21}, should be regarded as preliminary. The storage capacity of the receiving layer may become limiting. The storage capacity of the lower zone, also called the uptake capacity, is the amount of air plus the loss (van Diepen {\it et al}., 1988). The storage capacity can de defined as:

\begin{equation}
\label{eq:6.25}
UP  ~=~{\frac{(\, RD _{\max } \, -\, RD\, )\, \theta  _{\max } ~-~ W _{lz} }{\Delta t}} ~+~ Loss
\end{equation}

Where:\\[5pt]
\begin{tabularx}{\textwidth}{llXr}
	UP &:& Uptake capacity of lower zone  & [cm d$^{{\rm -1}}$]\\
	RD$_{{\rm max}}$ &:& Maximum rooting depth  & [cm]\\
	RD &:& Actual rooting depth  & [cm]\\
	W$_{{\rm lz}}$ &:& Amount of soil moisture in lower zone  & [cm]\\
	$\theta$$_{{\rm max}}$ &:& Soil porosity (maximum soil moisture)  & [cm$^{{\rm 3}}$ cm$^{{\rm -3}}$]\\
	$\Delta$t &:& Time step  & [d]\\
	Loss &:& Percolation rate from the lower zone to the subsoil   & [cm d$^{{\rm -1}}$]\\
\end{tabularx}

The percolation to the lower part of the potentially rootable zone can not exceed the
uptake capacity of the lower zone. Therefore the percolation rate is equal to the minimum
of the calculated percolation rate (eq. \ref{eq:6.21}) and the uptake.

\subsubsection{Preliminary calculation of the infiltration rate}
The infiltration rate depends on the amount of available water and the infiltration capacity
of the soil. If the actual surface storage is less than or equal to 0.1 cm, the preliminary
infiltration capacity is simply described as:

\begin{equation}
\label{eq:6.26}
IN_{p} =~ (\, 1\, -\, F _{I} \, C _{I} \, )\, P~+~ I _{e~} +~ \frac{SS_{t}}{ \Delta t} 
\end{equation}

Where:\\[5pt]
\begin{tabularx}{\textwidth}{llXr}
	IN$_{{\rm p}}$ &:& Preliminary infiltration rate  & [cm d$^{{\rm -1}}$]\\
	F$_{{\rm I}}$ &:& Maximum fraction of rain not infiltrating during time step t  & [-]\\
	C$_{{\rm I}}$ &:& Reduction factor applied to F$_{{\rm I}}$ as a function of the 
	precipitation intensity  & [-]\\
	P &:& Precipitation rate  & [cm d$^{{\rm -1}}$]\\
	I$_{{\rm e}}$ &:& Effective irrigation  & [cm d$^{{\rm -1}}$]\\
	SS$_{{\rm t}}$ &:& Surface storage at time step t (see eq. \ref{eq:6.29})  & [cm]\\
	$\Delta$t &:& Time step  & [d]\\
\end{tabularx}

The maximum fraction of rain not infiltrating during time step t, {\bf F$_{{\rm I}}$} 
(acronym: {\bf NOTINF})
can be either set to a fixed value or be made variable by multiplying F$_{{\rm I}}$ 
with a precipitation dependent reduction factor {\bf C$_{{\rm I}}$} (acronym: {\bf NINFTB}). 
If the fraction is variable, it means that it is maximum for high rainfall amounts, and 
that it will be reduced for low rainfall. The user should provide F$_{{\rm I}}$. 
Tabulated values of C$_{{\rm I}}$ are included in the model and
are assumed to be fixed. The infiltration rate calculated is preliminary, as the storage
capacity of the soil is not yet taken into account. 

If the actual surface storage is more than 0.1 cm, the available water which can 
potentially infiltrate, is equal to the amount of water, present on the surface, that is supplied via
rainfall and irrigation and depleted via evaporation from the water surface:

\begin{equation}
\label{eq:6.27}
IN_{p} ~=~P~+~I _{e} ~-~ E _{w~} +~ SS _{\frac{t}{\Delta t}} 
\end{equation}

Where:\\[5pt]
\begin{tabularx}{\textwidth}{llXr}
	IN$_{{\rm p}}$ &:& Preliminary infiltration rate  & [cm d$^{{\rm -1}}$]\\
	P &:& Precipitation intensity  & [cm d$^{{\rm -1}}$]\\
	I$_{{\rm e}}$ &:& Effective irrigation  & [cm d$^{{\rm -1}}$]\\
	E$_{{\rm w}}$ &:& Evaporation rate from a shaded water surface  & [cm d$^{{\rm -1}}$]\\
	SS$_{{\rm t}}$ &:& Surface storage at time step t (see eq. 6.29)  & [cm]\\
	$\Delta$t &:& Time step  & [d]\\
\end{tabularx}

However, the calculated infiltration rate is hampered by the conductivity of the soil. The calculated
infiltration rate cannot exceed the conductivity of the soil, which is used as an infiltration
limit. The conductivity of the soil (acronym: {\bf SOPE}) is soil specific and should be given
by the user.

\subsubsection{Adjusted infiltration}
The total loss of water from the root zone can now be calculated as the sum of the
transpiration, the evaporation and the percolation. The sum of this loss and the available
pore space in the root zone define the maximum infiltration rate. The 
infiltration rate cannot exceed this value. The maximum possible infiltration rate is given
by:

\begin{equation}
\label{eq:6.28}
IN_{\max } ~=~{\frac{(\, \theta  _{\max } \, -\, \theta  _{t} \, )\, RD}{\Delta t}} ~+~ T _{a} ~+~ E _{s} ~+~ Perc
\end{equation}

Where:\\[5pt]
\begin{tabularx}{\textwidth}{llXr}
	IN$_{{\rm max}}$ &:& Maximum infiltration rate  & [cm d$^{{\rm -1}}$]\\
	$\theta$$_{{\rm max}}$ &:& Soil porosity (maximum soil moisture)  
	& [cm$^{{\rm 3}}$ cm$^{{\rm -3}}$]\\
	$\theta$$_{{\rm t}}$ &:& Actual soil moisture content  
	& [cm$^{{\rm 3}}$ cm$^{{\rm -3}}$]\\
	RD &:& Actual rooting depth  & [cm]\\
	$\Delta$t &:& Time step  & [d]\\
	T$_{{\rm a}}$ &:& Actual transpiration rate   & [cm d$^{{\rm -1}}$]\\
	E$_{{\rm s}}$ &:& Evaporation rate from a shaded soil surface  & [cm d$^{{\rm -1}}$]\\
	Perc &:& Percolation rate from root zone to lower zone  & [cm d$^{{\rm -1}}$]\\
\end{tabularx}

Finally, the adjusted infiltration rate is calculated as:

\begin{equation}
\label{eq:INact}
IN ~=~ Min(IN_{p}, IN_{\max}, SOPE)
\end{equation}


\subsubsection{Surface runoff}

Surface runoff is taken into account by defining a maximum value for the surface storage. If the surface storage exceeds the maximum value for surface storage the exceeding amount of water will run off. The surface storage at time step t can be calculated as:

\begin{equation}
\label{eq:6.29}
SS_{t} ~=~ SS _{t-1} ~+~ (\, P ~+~ I _{e} ~-~ E _{w} ~-~ IN\, )\, \Delta t
\end{equation}

Where:\\[5pt]
\begin{tabularx}{\textwidth}{llXr}
	SS$_{{\rm t}}$ &:& Surface storage at time step t  & [cm d$^{{\rm -1}}$]\\
	P &:& Precipitation intensity  & [cm d$^{{\rm -1}}$]\\
	I$_{{\rm e}}$ &:& Effective irrigation rate  & [cm d$^{{\rm -1}}$]\\
	E$_{{\rm w}}$ &:& Evaporation rate from a shaded water surface  & [cm d$^{{\rm -1}}$]\\
	IN &:& Infiltration rate (adjusted)  & [cm d$^{{\rm -1}}$]\\
\end{tabularx}

The surface runoff can be calculated as:

\begin{equation}
%6.30
SR_{t} = SS_{t} - \min (SS_{t}, SS_{\max})\\
\end{equation}

Where:\\[5pt]
\begin{tabularx}{\textwidth}{llXr}
	SR$_{{\rm t}}$ &:& Surface runoff at time step t  & [cm]\\
	SS$_{{\rm t}}$ &:& Surface storage at time step t  & [cm]\\
	SS$_{{\rm max}}$ &:& Maximum surface storage  & [cm]\\
\end{tabularx}

{\bf SS$_{{\rm max}}$} (acronym: {\bf SSMAX}) is an environmental specific variable 
and should be provided by the user.

\subsubsection{Rates of change and root extension}
The rates of change in the amounts of water in the root zone and the lower zone are
calculated straightforward from the flows found above:

\stepcounter{equation}
\begin{align}
\label{eq:6.31}
\Delta W _{rz} &=~ (IN ~-~ T _{a} ~-~ E _{s} ~-~ Perc)\,\Delta \, t  \subeqn  \\
\Delta W _{lz} &=~ (Perc ~-~ Loss)\,\Delta \, t \subeqn
\end{align}

Where:\\[5pt]
\begin{tabularx}{\textwidth}{llXr}
	$\Delta$W$_{{\rm rz}}$ &:& Change of the amount of soil moisture in the root zone  & [cm]\\
	$\Delta$W$_{{\rm lz}}$ &:& Change of the amount of soil moisture in the lower zone  & [cm]\\
	T$_{{\rm a}}$ &:& Actual transpiration rate   & [cm d$^{{\rm -1}}$]\\
	E$_{{\rm s}}$ &:& Evaporation rate from a shaded soil surface  & [cm d$^{{\rm -1}}$]\\
	IN &:& Infiltration rate  & [cm d$^{{\rm -1}}$]\\
	Perc &:& Percolation rate from root zone to lower zone  & [cm d$^{{\rm -1}}$]\\
	Loss &:& Percolation rate from lower zone to sub soil  & [cm d$^{{\rm -1}}$]\\
	$\Delta$t &:& Time step  & [d]\\
\end{tabularx}

Due to extension of the roots into the lower zone, additional soil moisture becomes available.
The amount of the extra soil moisture in the new root zone and the amount of water in the
reduced lower zone can be calculated as:

\begin{equation}
\label{eq:6.32}
\Delta W_{rz} ~=~\Delta W_{lz} ~=~ -\, W_{lz} \,{\frac{RD_{t} \, -\, RD_{t-1} }{RD_{\max} \, -\, RD_{t-1}}}
\end{equation}

Where:\\[5pt]
\begin{tabularx}{\textwidth}{llXr}
	RD$_{{\rm t}}$ &:& Rooting depth at time step t  & [cm]\\
	RD$_{{\rm t-1}}$ &:& Rooting depth at time step t-1  & [cm]\\
	RD$_{{\rm max}}$ &:& Maximum rooting depth  & [cm]\\
	W$_{{\rm lz}}$ &:& Amount of soil moisture in the lower zone & [cm]\\
	$\Delta$W$_{{\rm rz}}$ &:& Change of the amount of soil moisture in the root zone  & [cm]\\
	$\Delta$W$_{{\rm lz}}$ &:& Change of the amount of soil moisture in the lower zone  & [cm]\\
\end{tabularx}

Note that equation ref{eq:6.32} will only be executed when there is a root growth ($RD_{t} \, -\, RD_{t-1} > 0$) avoiding a division by zero when the roots have reached their maximm depth.

With equation \ref{eq:6.31} the actual amount of water in the root zone and in the lower zone can
be calculated according to:

\stepcounter{equation}
\begin{align}
W _{rz,t} ~=~ W _{rz, t-1} ~+~ \Delta W _{rz} \subeqn  \\
W _{lz,t} ~=~ W _{lz, t-1} ~\, +~ \Delta W _{lz} \subeqn
\end{align}

Where:\\[5pt]
\begin{tabularx}{\textwidth}{llXr}
	W$_{{\rm rz,t}}$ &:& Amount of soil moisture in the root zone at time step t  & [cm]\\
	W$_{{\rm lz,t}}$ &:& Amount of soil moisture in the lower zone at time step t  & [cm]\\
	W$_{{\rm rz,t-1}}$ &:& Amount of soil moisture in the root zone at time step t-1  & [cm]\\
	W$_{{\rm lz,t-1}}$ &:& Amount of soil moisture in the lower zone at time step t-1  & [cm]\\
	$\Delta$W$_{{\rm rz}}$ &:& Rate of change of the amount of soil moisture in the root 
	zone  & [cm]\\
	$\Delta$W$_{{\rm lz}}$ &:& Rate of change of the amount of soil moisture in the 
	lower zone  & [cm]\\
\end{tabularx}

\subsubsection{Actual soil moisture content}
The actual soil moisture content can now be calculated according to (see also eq. \ref{eq:6.1}):

\begin{equation}
\label{eq:6.34}
\theta_{t} ~=~{\frac{ W_{rz, t}}{RD}} 
\end{equation}

Where:\\[5pt]
\begin{tabularx}{\textwidth}{llXr}
	$\theta$$_{{\rm t}}$ &:& Actual soil moisture content at time step t  & [cm$^{{\rm 3}}$ cm$^{{\rm -3}}$]\\
	W$_{{\rm rz,t}}$ &:& Amount of soil moisture in the root zone at time step t  & [cm]\\
	RD &:& Actual rooting depth  & [cm]\\
\end{tabularx}
%
%\subsection{The SWAP water balance}
%
%The WOFOST implementation connected to the SWAP model has a detailed water balance including 
%multiple soil layers and variable integration time steps to account for highly non-linear processes 
%in the soil. Besides estimating water availability, SWAP also deals with soil temperature and solute 
%transport which allows to make 
%detailed simulations of the behaviour of water and solutes in the soil and its impact on plant growth.
%The SWAP water balance is not treated in this manual as SWAP has its own set of documentation 
%\footnote{http://www.swap.alterra.nl} where the theory and usage of the model is described.
