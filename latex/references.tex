\chapter{References}

Acock, B., D.A. Charles-Edwards, D.J. Fitter, L.J. Ludwig, J. Warren Wilson \& A.C.
Withers, 1978. {\it The contribution of leaves from different levels within a tomato crop to
canopy net photosynthesis; an experimental examination of two canopy models\/}. Journal of
Experimental Botany 29:815-827.

Amthor, J.S., 1984. {\it The role of maintenance respiration in plant growth\/}. Plant, Cell and
Environment 7:561-569. 

Andrews, R.A. \& E.I. Newman, 1969. {\it Resistance to water flow in soil and plant. III.
Evidence from experiments with wheat\/}. New Phytologist 68:1051-1058.

\AA ngstr\"{o}m, A., 1924. {\it Solar and terrestrial radiation\/}. Quarterly Journal of the Royal
Meteorological Society 50:121-125.

Angus, J.F., R.B. Cunningham, M.W. Moncur \& D.H. Mackenzie, 1981. {\it Phasic
development in field crops\/}. I. Thermal response in seedling phase. Field Crops Research
3:365-378.

Bakel, P.J.T. van, 1981. {\it Unsaturated zone and evapotranspiration\/}. In: Evaporation in
relation to hydrology. Verslagen Mededelingen Commissie Hydrologische Onderzoekingen
T.N.O. 28, The Hague, The Netherlands.

Bakker, E.J., 1992. {\it Rainfall and risk in India's agriculture. An ex-ante evaluation of rainfall
insurance\/}. Groningen theses in economics, management \& organization. Wolters-Noordhoff,
Groningen. 180 pp.

Berge, ten H.F.M., 1986. {\it Heat and water transfer at the bare soil surface. Aspects affecting
thermal imagery\/}. Ph.D. Thesis Agricultural University Wageningen, The Netherlands.

Berkhout, J.A.A., J. Huijgen, S. Azzali \& M. Menenti, 1988. {\it MARS definition study.
Results of the preparatory phase. Main report\/}. Report 17. SC-DLO, Wageningen. 111 pp.

Bridge, D.W., 1976. {\it A simulation model approach for relating effective climate to winter
wheat yields on the Great Plains\/}. Agricultural Meteorology 17:185-194.

Brunt, D., 1932. {\it Notes on radiation in the atmosphere\/}. I. Quarterly Journal of the Royal
Meteorological Society 58:389-420.

Bussel, L. G. J., van. Stehfest, E., Siebert, S., Müller, C., \& Ewert, F. (2015). {\it Simulation of 
the phenological development of wheat and maize at the global scale}. Global ecology and biogeography, 
24(9), 1018-1029.

Causton, D.R. \& J.C. Venus, 1981. {\it The biometry of plant growth\/}. Edward Arnold, London.
307 pp.

Chouard, P., 1960. \textit{Vernalization and Its Relations to Dormancy}. Annual Review of Plant Physiology, vol. 11, no. 1, 
pp. 191–238. doi:10.1146/annurev.pp.11.060160.001203.

Choisnel, E., O. de Villele \& F. Lacroze, 1992. {\it Une approche de uniformis\'{e}e du calcul de
l'\'{e}vapotranspiration potentielle pour l'ensemble des pays de la Communaut\'{e} Europ\'{e}enne\/}.
Joint Research Centre, Commission of the European Communities, EUR 14223 FR,
Luxembourg, 176 pp.

Cowan, J.R., 1965. {\it Transport of water in the soil-plant-atmosphere system\/}. Journal of
Applied Ecology 2:221-239.

Diepen, C.A. van, J. Wolf, H. van Keulen \& C. Rappoldt, 1989. {\it WOFOST: a simulation
model of crop production.\/} Soil Use and Management 5:16$-$24. 

Diepen, C.A. van, C. Rappoldt, J. Wolf \& H. van Keulen, 1988. {\it Crop growth simulation
model WOFOST. Documentation version 4.1,\/} Centre for World Food Studies, Wageningen,
The Netherlands. 299 pp. 

Diepen, C.A. van, H. van Keulen, F.W.T. Penning de Vries, I.G.A.M. Noij \& J.
Goudriaan, 1987. {\it Simulated variabil\-ity of wheat and rice in current weather conditions and
in future weather when ambient CO$_{{\rm 2}}$ has doubled\/}. Simulation reports CABO-TT 14. CABO-DLO, WAU-TPE, Wageningen. 40 pp.

Denmead, O.T. \& R.H. Shaw, 1962. {\it Availability of soil water to plants as affected by soil
moisture conditions and meteorological conditions\/}. Agronomy Journal 54:385-389.

Dobben, W.H. van, 1962. {\it Influence of temperature and light conditions on dry matter
distribution, development rate and yield in arable crops\/}. Netherlands Journal of Agricultural
Science 10:377-389.

Doorenbos, J. \& A.H. Kassam, 1979. {\it Yield response to water\/}. FAO Irrigation Drainage
Paper 33, FAO, Rome, Italy. 193 pp.

Doorenbos, J., A.H. Kassam, C. Bentvelder \& G. Uittenbogaard, 1978. {\it Yield response to
water\/}. U.N. Economic Commission West Asia, Rome, Italy.

Doorenbos, J. \& W.O. Pruitt, 1977. {\it Guidelines for predicting crop water requirements\/}.
FAO Irrigation and Drainage paper 24, Rome, Italy.

Downes, R.W., 1970. {\it Effect of light intensity and leaf temperature on photosynthesis and
transpiration in wheat and sorghum\/}. Australian Journal of biological Sciences 23:775-782.

Downton, W.J.S., 1975. {\it The occurrence of C$_{{\rm 4}}$ photosynthesis among plants\/}. Photosynthetica
9:97-105.

Driessen, P.M., 1986. {\it The water balance of the soil\/}. In: H. van Keulen \& J. Wolf (Eds.).
Modelling of agricultural production: weather, soils and crop. Simulation Monographs,
Pudoc, Wageningen, The Netherlands. pp. 76-116.

Dwyer, L.M. \& D.W. Stewart, 1986. {\it Effect of leaf age and position on net photosynthetic
rates in maize (Zea mays L.)\/}. Agricultural and Forest Meteorology 37:29-46.

Ehleringer, J.R. \& R.W. Pearcy, 1983. {\it Variation in quantum yield for CO$_{{\rm 2}}$ uptake among
C$_{{\rm 3}}$ and C$_{{\rm 4}}$ plants\/}. Plant Physiology 73:555-559.

Ehleringer, J.R., 1978. {\it Implications of quantum yield differences on the distributions of C$_{{\rm 3}}$
and C$_{{\rm 4}}$ grasses\/}. Oecologia (Berl.) 31:255-267.

Farquhar, G.D., S van Caemmerer \& J.A. Berry, 1980. {\it A biochemical model of
photosynthetic CO$_{{\rm 2}}$ assimilation in leaves of C$_{{\rm 3}}$ species\/}. Planta 149:78-90.

Feddes, R.A., P.J. Kowalik, \& H. Zaradny, H., 1978. {\it Simulation of field water use and
crop yield\/}. Simulation Monographs. Pudoc, Wageningen, The Netherlands. 195 pp.

Fr\`{e}re, M., 1979. {\it A method for the practical application of the Penman formula for the
estimation of potential evapotranspiration and evaporation from a free water surface\/}. FAO,
AGP: Ecol/ 1979/1, Rome, Italy.

Fr\`{e}re, M. \& G.F. Popov, 1979. {\it Agrometeorological crop monitoring and forecasting\/}. FAO
plant production and protection paper 17. FAO, Rome, Italy.

Gallagher, J.N., 1979. {\it Field studies for cereal leaf growth. 1. Initiation and expansion in
relation to temperature and ontogeny\/}. Journal of experimental Botany 30:625-636.

Gardner, W.R., 1960. {\it Dynamic aspects of water availability to plants\/}. Soil Science 89:63-73.

Gollan, T., J.B. Passioura \& R. Maas, 1986. {\it Soil water status affects the stomatal
conductance of fully turgid wheat and sunflower leaves\/}. Australian Journal of Plant
Physiology 13:459-464.

Goudriaan, J., 1988. {\it The bare bones of leave angle distribution in radiation models for
canopy photosynthesis and energy exchange\/}. Agricultural and Forest Meteorology 43:155-169.

Goudriaan, J., 1986. {\it A simple and fast numerical method for the computation of daily totals
of crop photosynthesis.\/} Agricultural and Forest Meteorology 38:249$-$254. 

Goudriaan, J., 1982. {\it Some techniques in dynamic simulation\/}. In:  F.W.T. Penning de Vries
\& H.H. van Laar (Eds.). Simulation of plant growth and crop production. Simulation
Monographs, Pudoc, Wageningen, The Netherlands. pp. 66-84.

Goudriaan, J. \& H.H. van Laar, 1978. {\it Calculation of daily totals of the gross assimilation
of leaf canopies\/}. Netherlands Journal of Agricultural Science 26:373-382.

Goudriaan, J., 1977. {\it Crop micrometeorology: a simulation study\/}. Simulation Monographs.
Pudoc, Wageningen.

Hadley, P., E.H. Roberts, R.J. Summerfield \& F.R. Minchin, 1984. {\it Effects of temperature
and photoperiod on flowering in soya bean (Glycine max (L.) Merrill): a quantitative model\/}.
Annals of Botany 53:669-681.

Hargreaves, G.L., G.H. Hargreaves \& J.P. Riley, 1985. {\it Irrigation water requirement for
Senegal River Basin.\/} J. of Irrigation and Drainage Engineer\-ing, ASCE 111 (3): 265-275.

Heemst, H.D.J. van, 1986a. {\it The distribution of dry matter during growth of a potato crop\/}. Potato
Research 29:55-66. 

Heemst, H.D.J. van, 1986b. {\it Crop phenology and dry matter distribution\/}. In:  H. van Keulen \& J.
Wolf (Eds.). Modelling of agricultural production: soil, weather and crops. pp 13-60.

Hooijer, A.A. \& T. van der Wal, 1994. {\it CGMS version 3.1, user manual\/}. Technical document 15.1.
SC-DLO, Wageningen. 170 pp.

Hunt, E.R., J.A. Weber \& D.M. Gates, 1985. {\it Effects of nitrate application on Amaranthus powellii
Wats\/}. I. Changes in photosynthesis, growth rates, and leaf area. Plant Physiology 79:609-613.

Huygen, J., 1992. {\it SWACROP2, a quasi-two-dimensional crop growth \& soil water flow simulation
model. User's guide\/}. WAU, Dept. of water resources, SC-DLO, Wageningen. 58 pp.

Huygen (Ed.), 1990. {\it Simulation studies on the limitations to maize production in Zambia\/}. Report 27.
SC-DLO, {\nobreak}Wageningen. 99 pp.

Jackson, M.B. \& M.C. Drew, 1984. {\it Effects of flooding on growth and metabolism of herbaceous
plants\/}. In: T.T. Kozlowski (Ed.). Flooding and Plant growth. Academic Press, London, pp. 47-128.

Jansen, D.M. \& P. Gosseye, 1986. {\it Simulation of millet (Pennisetum americanum) as influenced by
water stress\/}. Simulation Reports CABO-TT no. 10, Centre for Agrobiological Research and
Department of Theoretical Production Ecology, Wageningen Agricultural University, 108 pp.

Janssen, B.H, F.C.T. Guilking, D. van der Eijk, E.M.A. Smaling, J. Wolf \& H. van Reuler, 1990.
{\it A system for quantitative fertility of tropical soils (QUEFTS)\/}. Geoderma, 46:299-318.

Jarvis, P.G., 1981. {\it Stomatal conductance, gaseous exchange and transpiration\/}. In: Plants and their
atmospheric environment. Blackwell Scientific Publications, Oxford, pp. 175-214.

Johnson, I.R. \& J.H.M. Thornley, 1983. {\it Vegetative crop growth model incorporating leaf area
expansion and senescence, and applied to grass\/}. Plant, Cell and Environmental 6:721-729.

Jong, J.B.R.M. de, 1980. {\it Een karakterisering van de zonnestralen in Nederland\/}. Doctoraalverslag
Vakgroep Fysische Aspecten van de Gebouwde Omgeving afd. Bouwkunde en Vakgroep Warmte-
en Stromingstechnieken afd. Werktuigbouwkunde. Techn. Univ. Eindhoven, The Netherlands. pp.
97+67.

Kase, M. \& J. Catsk\'{y}, 1984. {\it Maintenance and growth components of dark respiration rate in leaves
of C$_{{\rm 3}}$ and C$_{{\rm 4}}$ plants as affected by leaf temperature\/}. Biologia Plantarum 26:461-470.

Keulen, H. van, \& C.A. van Diepen, 1990. {\it Crop growth models and agro-ecological charac\-terizati\-on\/}. 
In: Scaife, A. (Ed.): Proceedings of the first con\-gress of the European so\-cie\-ty of agronomy, 5-7
December 1990, Paris. CEC, ESA, INRA. session 2:1-16.

Keulen, H. van, \& N.G. Seligman, 1987. {\it Simulation of water use, nitrogen nutrition and growth of
a spring wheat crop\/}. Simulation Monographs. Pudoc, Wageningen, The Netherlands. 310 pp.

Keulen, H. van, J. Wolf (Eds.), 1986. {\it Modelling of agricultural production: weather, soils and
crops.\/} Simulation Monographs, Pudoc, Wageningen, The Netherlands. 478 pp. 

Keulen, H. van, N.G. Seligman \& R.W. Benjamin, 1981. {\it Simula\-tion of water use and herbage
growth in arid\/} {\it regions - A re-evaluation and further development of the model `Arid Crop'\/}.
Agricultural systems 6:159-193.

Keulen, H. van, 1975. {\it Simulation of water use and herbage growth in arid regions\/}. Simulation
Monographs. Pudoc, Wageningen, The Netherlands. 184 pp.

Keulen, H. van \& C.G.E.M. van Beek, 1971. {\it Water movement in layered soils - A simulation model\/}.
Netherlands Journal of agricultural Science 19:138-153.

Kim, Dong-Hwan, Doyle, M.R., Sung, S., and Amasino, R.M. 2009.
\textit{Vernalization: Winter and the Timing of Flowering in Plants}. Annual Review of Cell and Developmental Biology, 
vol. 25, no. 1, pp. 277–99. doi:10.1146/annurev.cellbio.042308.113411.

Koning, G.H.J. de, H. van Keulen, R. Rabbinge \& H. Janssen, 1994. {\it Determination of input and
output coefficients of cropping systems in the European Community\/} (in prep.).

Koning, G.H.J. de, M.J.W. Jansen, E.R. Boons-Prins, C.A. van Diepen, F.W.T. Penning de Vries,
1993. {\it Crop growth simulation and statistical validation for regional yield forecasting across the
European Communities\/}. Joint Research Centre of the European Communities (JRC), Ispra, Italy.
Commission of the European Communities.

Koning, G.H.J. de, \& C.A. van Diepen, 1992. {\it Crop production potential of the rural areas within
the European Commun\-ities. IV: Potential, water-limited and actual crop production\/}. Technical
working document W68. Netherlands scientific council for government policy, The Hague. 83 pp.

Kraalingen, D.W.G. van, 1991. {\it The FSE system for crop simulation,\/} Simulation Reports CABO$-$TT
nr. 23, CABO$-$DLO, Wageningen Agricultural University, The Netherlands. 77 pp. 

Kraalingen, D.W.G \& C. Rappoldt, 1989. {\it Subprograms in simulation models\/}. Simulation Report
CABO-TT nr. 18. Centre for Agrobiological Research and Dept. of Theoretical Production Ecology,
Wageningen, The Netherlands. 54 pp.

Kropff, M.J., H.H. van Laar and H.F.M. ten Berge (Eds.), 1993. {\it ORYZA1 A basic model for
irrigated lowland rice production.\/} IRRI, Los Banos, The Philippines.

Laar, H.H. van, J. Goudriaan \& H. van Keulen (Eds.), 1992. {\it Simula\-tion of crop growth for
potential and water-limited production situations (as applied to spring wheat)\/}. Simulation reports
CABO$-$TT 27. CABO$-$DLO, WAU-TPE, {\nobreak}Wageningen. 72 pp.

Lanczos, C., 1957. {\it Applied Analysis\/}. Pitman, London.

Lanen, H.A.J. van, C.A. van Diepen, G.J. Reinds, G.H.J. de Koning, J.D. Bulens \& A.K. Bregt,
1992. {\it Physical land evaluation methods and GIS to explore the crop growth potential and its effects
within the European Commun\-ities\/}. Agricultural systems 39:307-328.

Leverenz, J.W. \& G. \"{O}quist, 1987. {\it Quantum yields of photosynthesis at temperatures between -2\degrees C
and 35\degrees C in a cold-tolerant C$_{{\rm 3}}$ plant (Pinus sylvestris) during the course of one year\/}. Plant, Cell and
Environment 10:287-95.

Loomis, R.S., R. Rabbinge \& E. Ng, 1979. {\it Explanatory models in crop physiology\/}. Annual Reviews
Plant Physiology 30:339-367.

Mellaart, E.A.R., 1989. {\it Toepassing van gewasgroei-{\nobreak}simulatiemodellen voor risico-studies in
sahellanden (The application of crop-growth simulation models for risk-studies in Sahelian countries)\/}.
In: Huijbers, C., S.P. Lingsma \& J.C. Oudkerk (Eds.) Informatica toepassingen in de agrarische
sector, voordrachten VIAS-Symposium 1989. 141-154.

Monteith, J.L., 1969. {\it Light interception and radiative exchange in crop stands\/}. In: J.D. Eastin, F.A.
Haskins, C.Y. Sullivan \& C.H.M. van Bavel (Eds.). Physiological aspects of crop yield. American
Society of Agronomy, Crop Science  Society of America, Madison, Wisconsin, U.S.A. pp. 89-111.

Netherlands scientific council for government policy, 1992. {\it Ground for choices, four perspectives
for the rural areas in the European Community\/}. Reports to the government 42. Sdu uitgeverij, The
Hague. 144 pp.

Newman, E.I., 1969a. {\it Resistance to water flow in soil and plant\/}. I. Soil resistance in relation to
amounts of roots: theoretical estimates. Journal of Applied Ecology 6:1-12.

Newman, E.I. 1969b. {\it Resistance to water flow in soil and plant\/}. II. A review of experimental
evidence on the rizosphere resistance. Journal of Applied Ecology 6:261-272.

Ng, E. \& R.S. Loomis, 1984. {\it Simulation of growth and yield the potato crop\/}. Simulation
Monographs, Pudoc, Wageningen, The Netherlands. 147 pp.

Nonhebel, S., 1994. {\it The effects of use of average instead of daily weather data in crop growth
simulation models\/}. Agricultural systems 44:377-396.

Parsons, A.J. \& M.J. Robson, 1981. {\it Seasonal changes in the  physiology of S24 perenial ryegrass
(Lolium perenne L.). 2. Potential leaf and canopy photosynthesis during the transition from vegetative
to reproductive growth\/}. Annals of Botany 47:249-258.

Peat, W.E., 1970, {\it Relationships between photosynthesis and light intensity in the tomato\/}. Ann. Bot.
34, 319-328.

Penman, H.L., 1948. {\it Natural evaporation from open water, bare soil and grass\/}. Proceedings Royal
Society, Series A 193:120-146.

Penman, H.L., 1956. {\it Evaporation: An introductory survey\/}. Netherlands Journal of Agricultural
Science 4:9-29.

Penning de Vries, F.W.T., D.M. Jansen, H.F.M. ten Berge \& A. Bakema, 1989. {\it Simulation of
ecophysiological processes of growth in several annual crops.\/} Pudoc, Wageningen, The Netherlands,
271 pp.

Penning de Vries, F.W.T. \& H.H. van Laar, 1982. {\it Simulation of growth processes and the model
BACROS. \/}In: Penning de Vries, F.W.T. \& H.H. van Laar (Eds.) Simulation of plant growth and
crop production. Simulation Monographs, Pudoc, Wageningen, The Netherlands. pp. 114-135.

Penning de Vries, F.W.T., J.M. Witlage \& D. Kremer, 1979. {\it Rates of respiration and of increase
in structural dry matter in young wheat, ryegrass and maize plants in relation to temperature, to
water stress and to their sugar content\/}. Annals of Botany (London) 44:595-609.

Penning de Vries, F.W.T., 1975. {\it The cost of maintenance processes in plant cells\/}. Annals of Botany
39:77-92.

Penning de Vries, F.W.T., A.H.M. Brunsting \& H.H. van Laar, 1974. {\it Products requirements and
efficiency of biosynthesis: a quantitative approach\/}. Journal of Theoretical Biology 45:339-377.

Pitter, R.L., 1977. {\it The effect of weather and technology on wheat yields in Oregon\/}. Agricultural
Meteorology 18:115-131.

Poels, R.L.H. \& W. Bijker, 1993. {\it TROPFOR, a computer pro\-gramme to simulate growth and water
use of tropical rain forests developed from the "WOFOST" programme\/}. WAU, Dept. of Soil science
and geology. 55 pp.

Pulles, J.H.M., J.H. Kauffman \& J. Wolf, 1991. {\it A user {\nobreak}friendly menu and batch facility for the crop
simulation model WOFOST v4.3. Supplement to WOFOST v4.1 User's Guide\/}. Technical paper.
International soil reference and information centre, Wageningen. 47 pp.

Rabbinge, R. \& C.T. de Wit, 1989. {\it Systems, models and simulation\/}. In: R. Rabbinge, S.A. Ward
\& H.H. van Laar (Eds.) Simulation and systems management in crop protection. Simulation
Monographs, Pudoc, Wageningen, The Netherlands. pp. 3-15.

Raghavendra, A.S. \& V.S.R. Das, 1978. {\it The occurrence of C$_{{\rm 4}}$ photosynthesis: a supplementary list
of C$_{{\rm 4}}$ plants reported during late 1974 -mid 1977\/}. Photosynthetica 12:200-208.

Rappoldt, C. \& D.W.G. van Kraalingen, 1990. {\it Reference manual of the FORTRAN utility library
TTUTIL with applications,\/} Simulation Reports CABO$-$TT nr. 20, CABO$-$DLO, Wageningen
Agricultural University, The Netherlands. 122 pp. 

Rawson, H.M., J.H. Hindmarsh, R.A. Fischer \& Y.M. Stockman, 1983. {\it Changes in leaf
photosynthesis with plant ontogeny and relationships with yield per ear in wheat cultivars and 120
progeny\/}. Australian Journal of Plant Physiology 10:503-514.

Reinink, K., I. Jorritsma \& A. Darwinkel, 1986. {\it Adaption of the AFRC wheat phenology model for
Dutch conditions\/}. Netherlands Journal of Agricultural Science 34:1-13.

Rietveld, J.J., 1978. {\it Soil non wettability and its relevance as a contributing factor to surface runoff
on sandy dune soils in Mali\/}. Internal Report Department of Theoretical Production Ecology,
Agricultural University, Wageningen. 179 pp.

Ritchie, J.R., 1972. {\it Model for predicting evaporation from a row crop with incomplete cover\/}. Water
Resources Research 8:1204-1213.

Ritchie, J.R., 1971. {\it Dryland evaporative flux in a subhumid climate\/}. II. Plant influences. Agronomy
Journal 63:56-62.

R\"{o}tter, R., 1993. {\it Simulation of the biophysical limitations to maize production under rainfed
conditions in Kenya. Evaluation and application of the model WOFOST\/}. {\nobreak}Materialien zur Ostafrika-Forschung, 
Heft 12. {\nobreak}Geographischen Gesellschaft Trier. 261 pp.

Ruijter, F.J. de, W.A.H. Rossing \& J. Schans. {\it Simulatie van opbrengstvorming bij tulp met
WOFOST\/}. Simula\-tion reports CABO$-$TT 33. CABO$-$DLO, WAU-TPE, Wageningen. 31 pp.

Salim, M.H., G.W. Todd \& A.M. Schlehuber, 1965. {\it Root development of wheat, oats and barley
under conditions of soil moisture stress\/}. Agronomy Journal 57:603-607.

Salisbury, F.B., 1981. {\it Responses to Photoperiod\/}. Encyclopedia Plant Physiology, New Series, Vol.
12A. Springer Verlag, Berlin, pp. 135-168.

Schapendonk, A.H.C.M. \& P. Gaastra, 1984. {\it A simulation study on CO$_{{\rm 2}}$ concentration in protected
cultivation\/}. Scientia Horticulturae 23:217-229.

Scheid F., 1968. {\it Theory and problems of numerical analysis.\/} New York, United States, 422 pp.

Schulze, E.D., 1986. {\it Carbon dioxide and water vapor exchange in response to drought in the
atmosphere and in the soil\/}. Annual Review of Plant Physiology 37:247-274.

Sheehy, J.M., J.M. Cobby \& G.J.A. Ryle, 1980. {\it The use of a model to investigate the influence of
some environmental factors on the growth of perennial ryegrass\/}. Annals of Botany 46:343-365.

Slayter, R.O. \& W.R. Gardner, 1965. {\it Overall aspects of water movement in plants and soils\/}.
Symposia of the Society for experimental Biology 19:113-129.

Spitters, C.J.T., H. van Keulen \& D.W.G. van Kraalingen, 1989. {\it A simple and universal crop
growth simulator: SUCROS87\/}. In: R. Rabbinge, S.A. Ward \& H.H. van Laar (Eds.) Simulation and
systems management in crop protection. Simulation Monographs, Pudoc, Wageningen, The
Netherlands. pp. 147-181.

Spitters, C.J.T., H.A.J.M. Toussaint, J. Goudriaan, 1986. {\it Separating the diffuse and direct
component of global radiation and its implications for modelling canopy photosynthesis.\/} Part I:
components of incoming radiation. Agricultural and Forest Meteorology 38:217$-$229. 

Spitters, C.J.T. \& Th. Kramer, 1986. {\it Differences between spring wheat cultivars in early growth\/}.
Euphytica 35:273-292.

Spitters, C.J.T., 1986. {\it Separating the diffuse and direct component of global radiation and its
implications for modelling canopy photosynthesis. Part II: Calculation of canopy photosynthesis.\/}
Agricultural and Forest Meteorology 38: 231$-$242. 

Stroosnijder, L., 1987. {\it Soil evaporation: test of a practical approach under semi-arid conditions\/}.
Netherlands Journal of Agricultural Science 35:417-426.\\
 Stroosnijder, L. 1982. {\it Simulation of the soil water balance\/}. In: F.W.T. Penning de Vries \& H.H.
van Laar (Eds.): Simulation of plant growth and crop production. Simulation Monographs, Pudoc,
Wageningen, The Netherlands. pp. 175-193.

Stroosnijder, L. \& D Kon\'{e}, 1982. {\it Le bilan d'eau du sol\/}. In: F.W.T. Penning de Vries \& M.A.
Djit\`{e}ye (Eds.). La productivit\'{e} des paturages sah\'{e}liens. Une \'{e}tude des sol, des v\'{e}g\'{e}tations et de
l'explotation de cette resource naturelle. Pudoc, Wageningen, The Netherlands. pp. 135-165.

Stroosnijder, L., van Keulen H. \& Vachaud G., 1972. {\it Water movement in layered soils. 2.
Experimental confirmation of a simulation model\/}. Netherlands Journal of Agricultural Science 20:67-72.

Summerfield, R.J. \& Roberts, E.H. 1987. {\it Effects of illuminance on flowering in long and short day
grain legumes: a reappraisal and unifying model\/}. In: J.G. Atherton (Ed.): Manipulation of flowering.
Butterworth, London, pp. 203-223.

Supit, I. 1994. {\it Global radiation. \/}Joint Research Centre, Commission of the European Communities.\\
EUR 1575 EN.

Thompson, L.M., 1969. {\it Weather and technology in the production of corn in the U.S. Corn Belt\/}.
Agronomy Journal 61:453-456.

Vergara, B.S. \& T.T. Chang, 1985. {\it The flowering response of the rice plant to photoperiod\/}.
International Rice Research Institute, Los Ba\~{n}os, Philippines.

Versteeg, M.N. \& H. van Keulen, 1986. {\it Potential crop production prediction by some simple
calculation methods, as compared with computer simulations\/}. Agricultural Systems 19:249-272.

Viehmeyer, F.J. \& A.H. Hendrickson, 1931. {\it The moisture equivalent as a measure of the field
capacity of soils\/}. Soil Science 32:181-193.

Wang, E., \& Engel, T. (1998). {\it Simulation of phenological development of wheat crops}. Agricultural systems, 58(1), 1-24.

Wardlaw, I.F., 1974. {\it Temperature control of translocation\/}. In Bieleski, R.L., A.R. Ferguson \&
M.M. Creswell (Eds.) Mechanics of regulation of plant growth. Bulletin 12, The Royal Society of
New Zealand, Wellington. pp. 533-538.

Weir, A.H., P.L. Bragg, J.R. Porter \& J.H. Rayner, 1984. {\it A winter wheat crop simulation model
without water and nutrient limitations\/}. Journal of Agricultural Science 102:371-382.

Whisler, F.D., B. Acock, D.N. Baker, R.E. Fye, H.F. Hodges, Lambert, J.R., H.E. Lemon, J.M.
Mckinion \& V.R. Reddy, 1986. {\it Crop simulation models in agronomic systems\/}. Advances in
Agronomy 401:141-208.

Williams, L.E., 1985. {\it Net photosynthetic rate and stomatal and intracellular conductances
subsequent to full leaf expansion in Zea mays L.: effect of leaf position\/}. Photosynthetica 19:397-401.

Wit, C.T. de, \& H. van Keulen, 1987. {\it Modelling production of field crops and its requirements\/}.
Geoderma 40:254-265.

 Wit, C.T. de, 1982. {\it Simulation of living systems\/}. In: F.W.T. Penning de Vries \& H.H. van Laar
(Eds.). Simulation of plant growth and crop production. Simulation Monographs, Pudoc,
Wageningen, The Netherlands. pp. 3-7.

Wit, C.T. de \& F.W.T. Penning de Vries, 1982. {\it L'analyse des syst\`{e}mes de production primaire\/}.
In: F.W.T. Penning de Vries \& M.A. Djiteye (Eds.). La Productivit\'{e} des p\^{a}turages sah\'{e}lens.
Agricultural Research Reports 918. Pudoc, Wageningen, The Netherlands. pp. 20-27.

Wit, C.T. et al., 1978. {\it Simulation of assimilation and transpiration of crops\/}. Simulation
Monographs, Pudoc, Wageningen, The Netherlands. 100 pp.

Wit, C.T. de \& J. Goudriaan, 1978. {\it Simulation of ecological processes\/}. Simulation Monographs,
Pudoc, Wageningen, The Netherlands. 175 pp.

Wit, C.T., 1965. {\it Photosynthesis of leaf canopies\/}. Agricultural Research Reports 663. Pudoc,
Wageningen, The Netherlands. 57 pp.

Wolf, J. \& C.A. van Diepen, 1991. {\it Effects of climate change on crop production in the Rhine basin\/}.
Report 52. RIZA, SC-DLO, Wageningen. 144 pp.

Wolf, J., 1993. {\it Effects of climate change on wheat and maize production potential in the EC\/}. In:
Kenny, G.J., P.A. Harrison \& M.L. Parry (Eds.). The effect of climate change on agricultural and
horticultural potential in Europe. Research report 2. Environmental change unit, University of
Oxford. pp. 93-119.

Wolf, J., J.A.A. Berkhout, C.A. van Diepen \& C.H. van {\nobreak}Immerzeel, 1989. {\it A study on the
limitations to maize production in Zambia using simulation models and a geo\-graphic information
system\/}. In: Bouma, J. \& A.K. Brecht (Eds.). Land qualities in space and time, proceedings of a
symposium organized by the International society of soil science (ISSS), Wageningen, the
Netherlands, 22-26 August 1988. Pudoc, Wageningen. 209-215.

Zel, H. van der, 1989. {\it Riego en la sierra, la experiencia de PRODERM\/}. PRODERM, Cusco. 108
pp.

\newpage
{\bf Extra Documentation}

Bouman, B.A.M., 1993. {\it ORYZA\_W Rice growth model for irrigated and water-limited conditions.\/}
DLO-Centre for Agrobiological Research, Wageningen, The Netherlands.

Danalatos, N.G., 1992. {\it Quantified analysis of selected land use systems in the Larissa region,
Greece.\/} Doctoral thesis, Agricultural University, Wageningen, The Netherlands, 370pp.

Guiking, I., in prep. {\it User manual for the crop growth simulation model WOFOST version 6.0.\/}
Technical Report for the Pilot Project for the Application of Remote Sensing to Agricultural
Statistics, Joint Research Centre of the E.C. The Winand Staring Centre, Wageningen, the
Netherlands.

Kraalingen, D.W.G., W. Stol, P.W.J. Uithol \& M.G.M. Verbeek, 1991. {\it User Manual of
CABO/TPE Weather System.\/} Internal Communication, CABO-DLO, Agricultural University,
Wageningen, The Netherlands. 28 pp.

Kropff, M.J., L. Bastiaans and J. Goudriaan, 1987. {\it Implications of improvements in modeling canopy
photosynthesis in SUCROS (a simple and universal crop growth simulator).\/} Netherlands Journal of
Agricultural Science 35 (1987) 192-194.

Laar, van H.H., J. Goudriaan \& H. van Keulen, 1992. {\it Simulation of crop growth for potential and
water limited production situations (as applied to spring wheat).\/} Department of Theoretical
Production Ecology (TPE-WAU) and DLO-Centre for Agrobiological Research (CABO-DLO),
Wageningen, The Netherlands. 78 pp.

Stol, W., H. van Keulen \& D.W.G. van Kraalingen, 1993. {\it The FORTRAN version of the Van Keulen
-Seligman CSMP-Spring wheat model.\/} CABO-DLO, Wageningen, The Netherlands.

Zande, J.C., 1991. {\it Overzicht van de structuur en benodigde invoer van enkele simulatie-modellen
voor de bodemwaterbalans en de gewasgroei.\/} IMAG, Wageningen, The Netherlands. 120 pp.

