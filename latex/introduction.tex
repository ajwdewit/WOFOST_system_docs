
\chapter{Introduction} 

\section{About WOFOST}

WOFOST is the acronym for WOrld FOod STudies. It is the name of a model for
simulating cropping systems involving the growth of crops in interaction
with its environment including weather, soil and agromanagement. 
WOFOST has been developed by
Wageningen University and Research (and its predecessors) already
since the 1980-ies. Currently, the model is maintained and improved by Wageningen 
University and Research in cooperation with the Joint Research Centre of the European 
Commission. Several implementation of the model are available which have been
developed for both research and operational applications.

for a general introduction to the WOFOST model and an overview of the
processes included in the model, see:\\
Wit, Allard de, Hendrik Boogaard, Davide Fumagalli, Sander Janssen, Rob Knapen, 
Daniel van Kraalingen, Iwan Supit, Raymond van der Wijngaart, and Kees van Diepen. 
\textit{25 Years of the WOFOST Cropping Systems Model.} Agricultural Systems 168 
(January 1, 2019): 154–67. \footnote{https://doi.org/10.1016/j.agsy.2018.06.018}


\section{Levels of crop production}

To be able to deal with the ecological diversity of agriculture, three hierarchical levels of 
crop growth can be distinguished: potential growth, limited growth and reduced growth. 
Each of these growth levels corresponds to a level of crop production: potential, limited and 
reduced production

Reality rarely corresponds exactly to one of these growth/production levels, but it is 
useful to reduce specific cases to one of them, because this enables you to focus on the 
principal environmental constraints to crop production, such as light, temperature, 
water and the macro nutrients nitrogen (N), phosphorus (P) and potassium (K). Other 
factors can often be neglected because they do not influence the crop's growth rate 
(De Wit, 1986; Rabbinge and De Wit, 1989; Penning de Vries et al., 1989).

{\it Potential production level\/}\\
Potential production represents the absolute production ceiling for a given crop when grown in 
a given area under specific weather conditions. It is determined by the crop’s photosynthesis
response to CO$_{2}$ 
and the temperature and solar radiation regimes during the growing season. In practice, this 
ceiling can only be reached with a high input of fertilizers, irrigation and thorough pest and
weed control. In addition, crop establishment should be perfect, there should be no losses 
caused by traffic or grazing, and there should be no damage to the crop by wind, hail, frosts.
Because potential yield is also determined by crop properties, yield potential varies over
crop varieties and can be increased by breeding. In the computer model, potential yield 
depends on the choice of crop variety, sowing date and weather data set. Near to potential 
yield levels are realized on some arable and grassland farms in Western Europe, and 
sometimes in (glasshouse) horticulture. In these cases, the weather conditions and 
crop characteristics exclusively determine the potential growth rate. When the canopy 
fully covers the soil, the increase in biomass, expressed in dry matter, is typically 
between 150 and 350 kg ha$^{-1}$ day$^{-1}$.

{\it Attainable production level\/}\\
At this production level the yield of the crop is limited by the availability of water and/or nutrients 
during a part or the complete growing season.  
In this scenario, the water-limited yield represents the maximum yield that can be obtained under 
rain-fed conditions but with optimal nutrient supply. The yield limiting effect of drought depends on 
the soil moisture availability as determined by the amounts of rainfall and evapotranspiration, and 
heir distribution over the growing season, by soil type, soil depth and groundwater influence. The 
difference between potential and water-limited production indicates the production increase that 
could be achieved by irrigation. A special case of water-limited conditions is related to an excess 
of soil moisture, causing oxygen shortage for the plant roots. So oxygen limited production would be 
a more appropriate term. Its effect depends on soil properties and drainage measures and is difficult 
to quantify by modelling.

The nutrient limited production corresponds to a situation that water is not limiting but when no 
manure or fertilizers are added to the soil. Then, nutrient availability depends on soil nutrient supply only
which is often insufficient and becomes limiting to the growth of the crop. 
Usually in (sub)-humid climates, the water-limited yield 
is higher than the nutrient-limited yield, so that yields can be increased by nutrient application without 
irrigation, but in cases of severe droughts the nutrient-limited yield may exceed the water-limited 
yield, and then water should be applied to increase yield.

{\it Reduced production level\/}\\
In this scenario the crop yield can be further reduced by factors such as pest, disease, competition with weeds
and pollutants (ozone, salt or heavy metals). This yield level reflects what farmers actual
harvest from their fields. The gaps between the different yield levels (yield gap) vary widely
across the globe and finding approaches to close the yield gap is an area of ongoing research,
see http://yieldgap.org

Note that the WOFOST 7.2 crop simulation model can be applied in the domain of potential crop
production and production with a water shortage (potential and water-limited yield levels). 
To estimate nutrient limited product the output from WOFOST has often been coupled to the QUEFTS
model (Jansen et al., 1990) which can estimate nutrient requirements and nutrient-limited yields.
However, QUEFTS is not a dynamic simulation model but uses seasonal nutrient supply to estimate
nutrient-limited yield. Therefore it is not regarded to be part of WOFOST.

\section{Guide to this manual}

This manual covers a detailed description of the processes of crop growth and water movement as they are
implemented in WOFOST Version 7.2. First of all, ancillary calculation will be described including:
\begin{itemize}
	\item derived meteorological variables
	\item reference evapotranspiration
	\item day length and solar elevation
	\item extra terrestrial radiation
\end{itemize}

Next, the different components of the crop simulation itself will be described in detail:
\begin{itemize}
	\item Phenology
	\item Transpiration
	\item Assimilation:
	\subitem Gross photosynthesis rate
	\subitem Correction for suboptimal temperature
	\subitem Correction for water stress
	\item Maintenance respiration
	\item Growth of the crop:
	\subitem Net photosynthesis rate
	\subitem Growth respiration
	\subitem Partitioning
	\subitem Leaf growth and senescence 
	\subitem Stems
	\subitem Roots
	\subitem Storage organs	
\end{itemize}

Finally, the soil components are described which consist of several water balance implementations.